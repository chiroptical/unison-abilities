%\PassOptionsToPackae{gray}{xcolor}
\documentclass[hyperref={pdfpagelabels=false},12pt]{beamer}
\setbeamertemplate{frametitle}[default][center]
\mode<presentation>
{
 \usetheme{Warsaw}      % or try Darmstadt, Madrid, Warsaw, ...
 \usecolortheme{default} % or try albatross, beaver, crane, ...
 \usefonttheme{default}  % or try serif, structurebold, ...
 \setbeamertemplate{footline}[frame number]
 \setbeamertemplate{caption}[numbered]
}

\usepackage[utf8]{inputenc}
\usepackage{helvet}
\usepackage{minted}
\usepackage{cancel}

% Great Commands
\newcommand{\ig}[2]{\includegraphics[width=#1\linewidth]{#2}}
\newcommand{\mybutton}[2]{\hyperlink{#1}{\beamerbutton{{#2}}}}
\newcommand{\myvbutton}[2]{\vfill\hyperlink{#1}{\beamerbutton{{#2}}}}
\newcommand{\code}[2]{\mintinline{#1}{#2}}
\newcommand{\python}[1]{\code{python}{#1}}
\newcommand{\haskell}[1]{\code{haskell}{#1}}
\newcommand{\txt}[1]{\code{text}{#1}}
\newcommand{\bash}[1]{\code{bash}{#1}}
\newcommand{\unnamedUrl}[1]{\href{#1}{\color{blue}{#1}}}
\newcommand{\namedUrl}[2]{\href{#1}{\color{blue}{#2}}}
\newcommand{\pygment}[3]{\inputminted[bgcolor=lightgray,linenos,fontsize=#1]{#2}{#3}}
\newcommand{\pygmentLines}[5]{\inputminted[bgcolor=lightgray,linenos,fontsize=#1,firstline=#2,lastline=#3,autogobble]{#4}{#5}}

% Color Scheme
\definecolor{pittblue}{RGB}{28,41,87}
\definecolor{pittgold}{RGB}{205,184,125}
\setbeamercolor{structure}{fg=pittgold}
\setbeamercolor{button}{bg=pittblue}

\title[Unison Abilities]{{Introduction to Unison Abilities}}
\author[Unison Abilities]{{Barry Moore II}}
\institute[CRC]{Center for Research Computing \\ University of Pittsburgh}
\date{}

\beamertemplatenavigationsymbolsempty

\begin{document}

\begin{frame}{Getting the Slides}
  \begin{itemize}
    \item Download slides from
      \namedUrl{https://github.com/chiroptical/unison-abilities/releases/latest/download/unison-abilities.pdf}{https://tinyurl.com/unisonabilities}
    \item \unnamedUrl{https://github.com/chiroptical/unison-abilities}
  \end{itemize}
\end{frame}

\begin{frame}[plain]
\titlepage
\end{frame}

\begin{frame}{Outline}
  \begin{itemize}
    \item Choose an effect to encode
    \item Free monad encoding
    \item Tagless final encoding
    \item Encoding using Unison abilities
  \end{itemize}
\end{frame}

\begin{frame}{The Effect in Question}
  \begin{itemize}
    \item Imagine a \haskell{Stack} effect with the following operations:
      \begin{itemize}
          \item \haskell{get} the current stack
          \item \haskell{put} a value on top of the stack
          \item \haskell{pop} the top from the stack (if any exist)
      \end{itemize}
  \end{itemize}
\end{frame}

\begin{frame}{Haskell code is just a \bash{stack} script}
  \pygmentLines{\scriptsize}{1}{2}{haskell}{code/Free.hs}
  \begin{itemize}
    \item Run them with \bash{stack <filename>}
  \end{itemize}
\end{frame}

\begin{frame}{Encoding with Free Monads (Haskell)}
  \pygmentLines{\normalsize}{17}{21}{haskell}{code/Free.hs}
  \vspace{-0.5cm}
  \begin{itemize}
    \item Operation's return type on the left of \haskell{->}, \haskell{()} for
      \haskell{Put}
    \item You get a Monad for ``free'' by deriving \haskell{Functor} and
      using the \haskell{Free} type class
  \end{itemize}
  \pygmentLines{\normalsize}{23}{23}{haskell}{code/Free.hs}
\end{frame}

\begin{frame}{Smart Constructors}
  \begin{itemize}
    \item Define ``smart constructors'' to lift your operations into \haskell{Free}
  \end{itemize}
  \pygmentLines{\normalsize}{25}{32}{haskell}{code/Free.hs}
\end{frame}

\begin{frame}{The Interpreter}
  \pygmentLines{\scriptsize}{10}{11}{haskell}{code/Free.hs}
  \vspace{-1cm}
  \pygmentLines{\normalsize}{34}{40}{haskell}{code/Free.hs}
  \vspace{-0.5cm}
  \begin{itemize}
    \item Natural transformation required to interpret the \haskell{Stack}
    \item \haskell{Free} is a Monad, so we can use \haskell{>>=, >>}
    \item \haskell{iterM} evaluates the functorial context away
    \item \haskell{k} is a continuation
  \end{itemize}
\end{frame}

\begin{frame}{Encoding programs with SCs}
  \pygmentLines{\scriptsize}{42}{56}{haskell}{code/Free.hs}
  \vspace{-0.5cm}
  \begin{itemize}
    \item \haskell{one} -- \haskell{[2, 1]}
    \item \haskell{two} -- \haskell{[6]}
    \item \haskell{three} -- \haskell{Nothing}
    \item Each program returns a \haskell{State} computation
  \end{itemize}
\end{frame}

\begin{frame}{Running a \haskell{State} computation}
  \pygmentLines{\normalsize}{58}{62}{haskell}{code/Free.hs}
  \begin{itemize}
    \item Let's check out \bash{Free.hs}
  \end{itemize}
\end{frame}

\begin{frame}{Encoding using tagless final style}
  \begin{itemize}
    \item Define a type class which abstracts over a representation (i.e. \haskell{State [a]})
  \end{itemize}
  \pygmentLines{\normalsize}{14}{17}{haskell}{code/TaglessFinal.hs}
\end{frame}

\begin{frame}{Helper functions}
  \pygmentLines{\normalsize}{26}{33}{haskell}{code/TaglessFinal.hs}
  \begin{itemize}
    \item We don't need to lift our primitives, but the compiler needs help
      with type inference
  \end{itemize}
\end{frame}

\begin{frame}{Instance declaration}
  \pygmentLines{\normalsize}{12}{12}{haskell}{code/TaglessFinal.hs}
  \vspace{-1cm}
  \pygmentLines{\normalsize}{19}{24}{haskell}{code/TaglessFinal.hs}
  \vspace{-0.5cm}
  \begin{itemize}
    \item The implementation looks similar to free monad style
    \item No continuation makes the code look a bit simpler
  \end{itemize}
\end{frame}

\begin{frame}{Programs}
  \pygmentLines{\scriptsize}{35}{49}{haskell}{code/TaglessFinal.hs}
  \vspace{-0.5cm}
  \begin{itemize}
    \item This should look familiar!
  \end{itemize}
\end{frame}

\begin{frame}{Running}
  \pygmentLines{\scriptsize}{51}{57}{haskell}{code/TaglessFinal.hs}
  \begin{itemize}
    \item Let's check out \bash{TaglessFinal.hs}
  \end{itemize}
\end{frame}

\begin{frame}{Free Monad vs. Tagless Final}
  \begin{itemize}
    \item Program implementations stayed the same
    \item Free monad style requires a continuation
    \item Tagless final style has some type inference issues in this case
  \end{itemize}
\end{frame}

\begin{frame}{Unison Abilities}
  \begin{itemize}
    \item The abilities system in Unison is extensible, similar to polysemy,
      fused-effects, or eff
    \item The encoding of \haskell{Stack} in abilities looks a lot like tagless
      final style
    \item The implementation is based on a language called Frank, see
      \unnamedUrl{https://arxiv.org/pdf/1611.09259.pdf}
  \end{itemize}
\end{frame}

\begin{frame}{The \textit{ability}}
  \pygmentLines{\scriptsize}{42}{45}{haskell}{code/ability.u}
  \begin{itemize}
    \item We don't need the \haskell{repr} from before!
  \end{itemize}
\end{frame}

\begin{frame}{Concepts}
  \begin{itemize}
    \item Delayed computation: \txt{'a} $\sim$ \txt{() -> a}
    \begin{itemize}
      \item Can be forced with \txt{!}
    \end{itemize}
    \item A \txt{Request {Ability a} b} can be,
      \begin{itemize}
        \item An ability member and a continuation e.g. \txt{{Ability.member -> k}}
        \begin{itemize}
          \item The input to \txt{k} must match the return type of
            \txt{Ability.member}
        \end{itemize}
        \item A pure value e.g. \txt{{x}} with \txt{x : b}
      \end{itemize}
    \item A recursive function \txt{h} is typically used to evaluate programs
      to their \textit{pure} values
    \item \txt{h}'s first argument will simply be an accumulator
    \item Handling requests: \txt{handle k _ with h _}, the input to
      \txt{h} is the new state
  \end{itemize}
\end{frame}

\begin{frame}{Interpreter}
  \pygmentLines{\footnotesize}{11}{21}{text}{code/ability.u}
\end{frame}

\begin{frame}{Programs}
  \pygmentLines{\footnotesize}{23}{40}{text}{code/ability.u}
\end{frame}

\begin{frame}{Why are these delayed?}
  \begin{itemize}
    \item Abilities can not be attached to values, i.e. \xcancel{\txt{{Ability a} b}}
    \item Abilities must be on the right of an arrow
    \item We must delay the computation to use the ability
    \begin{itemize}
      \item \txt{'let} is used in front of the block
    \end{itemize}
  \end{itemize}
\end{frame}

\begin{frame}{Running}
  \pygmentLines{\footnotesize}{2}{4}{text}{code/ability.u}
  \begin{itemize}
    \item Let's check out \bash{ability.u}
  \end{itemize}
\end{frame}

\begin{frame}{Questions}
  \Huge Questions?
\end{frame}

\end{document}
